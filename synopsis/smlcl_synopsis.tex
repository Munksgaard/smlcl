\documentclass[a4paper, 10pt]{article}

\usepackage[utf8x]{inputenc}
\usepackage[english]{babel}
\usepackage{fancyhdr}
\usepackage{hyperref}
\usepackage{url}
\usepackage{graphicx}
\usepackage{ku-forside}
\usepackage{xcolor, colortbl}
\usepackage{cite}
\usepackage{url, hyperref}
\usepackage{amsmath}

%\setlength\parskip{1em}
%\setlength{\parindent}{0pt}
\def\arraystretch{2}

\fancyhead[LO,RE]{smlcl \\ }
\fancyhead[LE,RO]{Philip Munksgaard \\ \today}
\pagestyle{fancy}

\titel{smlcl}
\undertitel{An ML library for utilizing parallel architectures using OpenCL}
\opgave{Bachelorproject}
\forfatter{\shortstack[l]{P. Munksgaard (240789)}}
\dato{\today}
\vejleder{M. Elsman}

\begin{document}
\maketitle

\newpage

\tableofcontents

\newpage

\section{Introduction}

TODO

\section{Problem description}

In this project \emph{I want to examine whether or not it is possible to
construct a Standard ML library for utilizing parallel architectures
using OpenCL, that will enable users to achieve significant speedups
in potentially parallel computations compared to traditional
sequential architectures}.

\section{Background}

OpenCL is a specification for a system that allows developers to take
advantage of parallel architectures such as graphics cards or
multiple-core CPUs in a unified manner, thus achieving improved
performance on parallel computations such as mapping over lists,
matrix manipulation, graphics rendering and more. Many different
vendors have implemented standard for their parallel architectures
such as Nvidia, Intel, AMD, and ATI.

The standard specification uses ``kernels'' written in a strict subset
of C, and ``hosts'' written in \verb+C+/\verb-C++-, that interface
with these kernels, to set up and run computations in parallel. We
want to construct a library for Standard ML that makes easy it do
parallel computations with OpenCL in a simple unified way. The goal is
to abstract away many of the technicalities and specifics of dealing
with OpenCL, and provide a simple and straightforward interface hiding
some of the complexities of the underlying system.

Furthermore, the project will include development of some illustrative
examples that uses this library to perform parallel computations and a
comparison to sequential architectures using these examples.

\section{Limitations}

The project will not include creation of user guides or structured
testing of the developed software beyond some basic examples as
described above.

The library will primarily be developed on a 64 bit machine equipped
with an Nvidia graphics card running Linux. The project will therefore
not include ensuring that the library runs equally well on other
operating systems or architectures.

\section{Risks}

TODO

\section{Project timeline}

\subsection{Execution of OpenCL kernels using a simplified API}
\label{sec:clibrary}

\begin{description}
  \item[Product] \hfill \\
    C library that allows for execution of OpenCL kernels using a
    simplified API.
  \item[Resources] \hfill \\
    None.
  \item[Internal dependencies] \hfill \\
    None.
  \item[Time requirement] \hfill \\
    2-3 days of work.
  \item[Deadline] \hfill \\
    Sunday, March 10th.
\end{description}

\subsection{Execution of OpenCL kernels from Moscow ML}
\label{sec:mosmlffi}

\begin{description}
  \item[Product] \hfill \\
    Moscow ML foreign function interface that allows for basic
    interaction with the above C library and execution of kernels
    passed as strings.
  \item[Resources] \hfill \\
    None.
  \item[Internal dependencies] \hfill \\
    The above mentioned C library.
  \item[Time requirement] \hfill \\
    3-4 days of work.
  \item[Deadline] \hfill \\
    Wednesday, April 17th.
\end{description}

\subsection{Parallel computations using the MosML FFI}

\begin{description}
  \item[Product] \hfill \\
    2-3 examples written for the above Moscow ML foreign function
    interface, utilizing the parallel architecture using OpenCL.
  \item[Resources] \hfill \\
    None.
  \item[Internal dependencies] \hfill \\
    The C library and the foreign function interface for Moscow ML.
  \item[Time requirement] \hfill \\
    1-2 days of work.
  \item[Deadline] \hfill \\
    Sunday, April 21st.
\end{description}

\subsection{Standard ML library for converting abstract data into
  OpenCL kernels}

\begin{description}
  \item[Product] \hfill \\
    A Standard ML library, smlcl, that abstracts away the complexities of the
    foreign function interface described above, and allows for kernels
    to be expressed using abstract data structures.
  \item[Resources] \hfill \\
    None.
  \item[Internal dependencies] \hfill \\
    Foreign function interface for MosML
  \item[Time requirement] \hfill \\
    6-7 days of work.
  \item[Deadline] \hfill \\
    Sunday, May 12th.
\end{description}

\subsection{Porting examples to Standard ML using abstract data structures}

\begin{description}
  \item[Product] \hfill \\ A port of the above examples, this time
    written purely in Standard ML, using the library above with
    abstract data structures representing parallel computations.
  \item[Resources] \hfill \\
    None.
  \item[Internal dependencies] \hfill \\
    smlcl
  \item[Time requirement] \hfill \\
    1-2 days of work.
  \item[Deadline] \hfill \\
    Friday, May 17th.
\end{description}

\subsection{Compare performance of smlcl with traditional sequential
  computation}

\begin{description}
  \item[Product] \hfill \\ A comparison between the performance of
    smlcl and traditional sequential computation.
  \item[Resources] \hfill \\
    None.
  \item[Internal dependencies] \hfill \\
    smlcl
  \item[Time requirement] \hfill \\
    2-3 days of work.
  \item[Deadline] \hfill \\
    Wednesday, May 22nd.
\end{description}


\subsection{Report}

\begin{description}
  \item[Product] \hfill \\ The finished report on the project
  \item[Resources] \hfill \\
    None.
  \item[Internal dependencies] \hfill \\
    All the above.
  \item[Time requirement] \hfill \\
    5-6
  \item[Deadline] \hfill \\
    Monday, June 3rd.
\end{description}

\section{Evaluations}

TODO

\section{References}

\begin{itemize}
  \item hopencl - \url{https://github.com/HIPERFIT/hopencl} \\
  \item PyOpenCL - \url{http://mathema.tician.de/software/pyopencl} \\
  \item OpenCL specification -
    \url{http://www.khronos.org/registry/cl/specs/opencl-1.2.pdf} \\
  \item OpenCL for Nvidia - \url{https://developer.nvidia.com/opencl}
    \\
\end{itemize}

\end{document}
