\documentclass[a4paper, 10pt]{article}

\usepackage[utf8x]{inputenc}
\usepackage[english]{babel}
\usepackage{fancyhdr}
\usepackage{hyperref}
\usepackage{url}
\usepackage{graphicx}
\usepackage{ku-forside}
\usepackage{xcolor, colortbl}
\usepackage{cite}
\usepackage{url, hyperref}
\usepackage{amsmath}

%\setlength\parskip{1em}
%\setlength{\parindent}{0pt}
\def\arraystretch{2}

\fancyhead[LO,RE]{smlcl \\ }
\fancyhead[LE,RO]{Philip Munksgaard \\ \today}
\pagestyle{fancy}

\titel{smlcl}
\undertitel{An ML library for utilizing parallel architectures using OpenCL}
\opgave{Bachelorproject}
\forfatter{\shortstack[l]{P. Munksgaard (240789)}}
\dato{\today}
\vejleder{M. Elsman}

\begin{document}
\maketitle

\newpage

\tableofcontents

\newpage

\section{Introduction}

In recent years, performing parallel computations using graphics cards
and other parallel architectures have become increasingly easy using
such frameworks as CUDA and OpenCL. Parallel computations offer
massive performance gains for certain kinds of computation, and is
already a crucial part of software development and computer science.

However, programming for OpenCL is done through a combination of C/C++
libraries and OpenCLs own language for specifying kernels. We wish to
make it easy to utilize OpenCLs capabilities through Standard ML.

\section{Problem description}

In this project \emph{I want to examine whether or not it is possible
  to construct a Standard ML library for utilizing parallel
  architectures using OpenCL, that will enable users to achieve
  significant speedups in potentially parallel computations compared
  to traditional sequential architectures}.

\section{Learning Objectives}

The goals for this project are

\begin{itemize}
  \item To describe the OpenCL framework for working with a CPU.
  \item To analyze the needs for a OpenCL library in MosML.
  \item Design and implement a simplified C library for OpenCL based
    on the identified needs, and develop a foreign function interface
    from Moscow ML to this C library.
  \item Design and develop an abstraction for working with parallel
    computations in SML on top of this foreign function interface
  \item Assert that the implementations work as expected through
    unit-testing and example code.
  \item Analyse and discuss performance of smlcl compared to
    sequential execution.
  \item Discuss possible extensions for the library.
\end{itemize}

\section{Background}

OpenCL is a cross-platform open standard that enables developers to
take advantage of parallel architectures such as graphics cards or
multiple-core CPUs in a unified manner, thus achieving improved
performance on parallel computations such as mapping over lists,
matrix manipulation, graphics rendering and more. Many different
vendors have implemented the standard for their parallel architectures,
such as Nvidia, Intel, AMD, and ATI.

The standard specification uses ``kernels'' written in a strict subset
of C, and ``hosts'' written in C/C++, that interface
with these kernels to set up and run computations in parallel. We
want to construct a library for Standard ML that makes easy it perform
parallel computations with OpenCL in a simple unified way. The goal is
to abstract away many of the technicalities and specifics of dealing
with OpenCL, and provide a simple and straightforward interface hiding
many of the complexities of the underlying system.

Furthermore, the project will include development of some illustrative
examples that uses this library to perform parallel computations and a
comparison to sequential architectures using these examples.

\section{Scope}

The project will not include creation of user guides or similar for
the library beyond the different examples as described above.

The library will primarily be developed on a 64 bit machine equipped
with an Nvidia graphics card running Linux. The project will therefore
not include ensuring that the library runs equally well on other
operating systems or architectures.

\section{Risks}

One important risk is that doing a decent translation from data
structure in SML to a kernel in OpenCLs language will prove a big
difficulty and take a lot of time.

Furthermore, OpenCL has a lot of complexity and internal choices that
have to be made when you're running a program, and it might be
difficult to properly take advantage of this and model the parallel
computations in such a way that significant performance gains are
possible, without exposing too much of the internal workings of OpenCL
to the user.

%% Lastly, the foreign function interface from Moscow ML to OpenCL might
%% prove a challenge, as there is a significant lack of documentation for
%% Moscow ML, and since it hasn't been actively developed for a number of
%% years.

\section{Project timeline}

\subsection{Execution of OpenCL kernels using a simplified API}
\label{sec:clibrary}

\begin{description}
  \item[Product] \hfill \\
    C library that allows for execution of OpenCL kernels using a
    simplified API.
  \item[Resources] \hfill \\
    None.
  \item[Internal dependencies] \hfill \\
    None.
  \item[Time requirement] \hfill \\
    2-3 days of work.
  \item[Deadline] \hfill \\
    Sunday, March 10th.
\end{description}

\subsection{Execution of OpenCL kernels from Moscow ML}
\label{sec:mosmlffi}

\begin{description}
  \item[Product] \hfill \\
    Moscow ML foreign function interface that allows for basic
    interaction with the above C library and execution of kernels
    passed as strings.
  \item[Resources] \hfill \\
    None.
  \item[Internal dependencies] \hfill \\
    The above mentioned C library.
  \item[Time requirement] \hfill \\
    3-4 days of work.
  \item[Deadline] \hfill \\
    Wednesday, April 17th.
\end{description}

\subsection{Parallel computations using the MosML FFI}

\begin{description}
  \item[Product] \hfill \\
    2-3 examples written for the above Moscow ML foreign function
    interface, utilizing the parallel architecture using OpenCL.
  \item[Resources] \hfill \\
    None.
  \item[Internal dependencies] \hfill \\
    The C library and the foreign function interface for Moscow ML.
  \item[Time requirement] \hfill \\
    1-2 days of work.
  \item[Deadline] \hfill \\
    Sunday, April 21st.
\end{description}

\subsection{Standard ML library for converting abstract data into
  OpenCL kernels}

\begin{description}
  \item[Product] \hfill \\
    A Standard ML library, smlcl, that abstracts away the complexities of the
    foreign function interface described above, and allows for kernels
    to be expressed using abstract data structures.
  \item[Resources] \hfill \\
    None.
  \item[Internal dependencies] \hfill \\
    Foreign function interface for MosML
  \item[Time requirement] \hfill \\
    6-7 days of work.
  \item[Deadline] \hfill \\
    Sunday, May 12th.
\end{description}

\subsection{Porting examples to Standard ML using abstract data structures}

\begin{description}
  \item[Product] \hfill \\ A port of the above examples, this time
    written purely in Standard ML, using the library above with
    abstract data structures representing parallel computations.
  \item[Resources] \hfill \\
    None.
  \item[Internal dependencies] \hfill \\
    smlcl
  \item[Time requirement] \hfill \\
    1-2 days of work.
  \item[Deadline] \hfill \\
    Friday, May 17th.
\end{description}

\subsection{Compare performance of smlcl with traditional sequential
  computation}

\begin{description}
  \item[Product] \hfill \\ A comparison between the performance of
    smlcl and traditional sequential computation.
  \item[Resources] \hfill \\
    None.
  \item[Internal dependencies] \hfill \\
    smlcl
  \item[Time requirement] \hfill \\
    2-3 days of work.
  \item[Deadline] \hfill \\
    Wednesday, May 22nd.
\end{description}


\subsection{Report}

\begin{description}
  \item[Product] \hfill \\ The finished report on the project
  \item[Resources] \hfill \\
    None.
  \item[Internal dependencies] \hfill \\
    All the above.
  \item[Time requirement] \hfill \\
    5-6
  \item[Deadline] \hfill \\
    Monday, June 3rd.
\end{description}

\section{Evaluation}

The project should be evaluated based on whether or not it was
possible to achieve its goals of interfacing with parallel
architectures from Standard ML using OpenCL. A successful
implementation of smlcl should make it possible to perform
computations in parallel that is significantly faster than performing
the same computation in a sequential manner. It should also make it
easy to express such computations using Standard ML data structures,
thus abstracting away many of the complexities of OpenCL, including
the kernels.

\section{References}

\begin{itemize}
  \item hopencl - \url{https://github.com/HIPERFIT/hopencl} \\
  \item PyOpenCL - \url{http://mathema.tician.de/software/pyopencl} \\
  \item OpenCL specification -
    \url{http://www.khronos.org/registry/cl/specs/opencl-1.2.pdf} \\
  \item OpenCL for Nvidia - \url{https://developer.nvidia.com/opencl}
    \\
  \item Moscow ML Library Documentation - \url{http://www.itu.dk/~sestoft/mosmllib/index.html} \\
  \item Moscow ML Owner's Manual - \url{http://www.itu.dk/people/sestoft/mosml/manual.pdf} \\
\end{itemize}

\end{document}
