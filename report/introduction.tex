\section{Introduction}

In this paper, we present SmlCL, a Standard ML module for performing
parallel computations on graphics cards via OpenCL.

In recent years, performing parallel computations using graphics cards
and other parallel architectures, has become increasingly easy using
such frameworks as NVIDIA's CUDA and the cross-platform framework
OpenCL. Parallel computations offer massive performance gains for
certain kinds of computation, and is already a crucial part of
software development and computer science.

However, programming for OpenCL is done through a combination of C/C++
libraries and OpenCLs own language for specifying kernels, in a very
involved manner that is prone to programming errors. We wish to make
it easy to utilize OpenCL's capabilities through Standard ML in an
easy-to-use, type safe manner.

We achieve this by creating a series of abstraction layers on top of
OpenCL: First a simplified C-interface to OpenCL called SimpleCL,
which we use to hide away many of the complexities and intricacies of
OpenCL, followed by a simple MLton interface to SimpleCL called
PrimCL, on top of which we then build SmlCL, to provide type safety
and the ability to safely generate kernels.

In this paper, we will first go through a short description of GPU
programming and parallelism, as well as a short introduction to how
OpenCL works. Then we will describe go on to describe the abstraction
layers that has been built on top of OpenCL, starting with SimpleCL,
continuing with PrimCL, followed by a preliminary discussion of
SmlCL. We will then go on to discuss type safety, the benefits thereof,
and how phantom types allow us to construct type safe OpenCL kernels
in Standard ML, and a description of how the SmlCL module was extended
with this functionality. Finally, we will discuss possible additions and
extensions to my work as well as related work, followed by a
conclusion.
