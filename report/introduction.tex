\section{Introduction}

In this paper, we present SmlCL, a Standard ML module for performing
parallel computations on graphical processing units via OpenCL.

In recent years, performing parallel computations using graphics cards
and other parallel architectures, has become increasingly easy using
such frameworks as NVIDIA's CUDA and the cross-platform framework
OpenCL. Parallel computations offer massive performance gains for
certain kinds of computation, and is already a crucial component of
software development and computer science.

However, programm for parallel architectures such as graphical
processing units (GPUs) is usually done at quite a low level. For
example, programming for OpenCL is done through a combination of C/C++
APIs and OpenCL's own C-like language for specifying kernels. The
process of writing kernels is very susceptible to forgetfulness and
errors from the programmers side, and the environment provides little
to no error-handling. We wish to make it easy to utilize parallel
architectures through Standard ML, by interfacing with OpenCL in a
safe and easy-to-use manner.

We achieve this by creating a series of abstraction layers on top of
OpenCL: First a simplified C-interface to OpenCL called SimpleCL,
which we use to hide many of the complexities and intricacies of
OpenCL, followed by a simple Standard ML interface written for
MLton~\cite{mlton} called PrimCL, that interfaces with
SimpleCL. Lastly, we build SmlCL on top of PrimCL, giving us the
ability to interface with GPUs through SML, while taking advantage of
static type-checking and high-level abstractions in order to generate
and execute type-safe kernels.

In this paper, we will first go through a short description of GPU
programming and parallelism, as well as a short introduction to how
OpenCL works. Then we will go on to describe the abstraction layers
that has been built on top of OpenCL, starting with SimpleCL and
continuing with PrimCL. We will then go on to discuss type safety, the
benefits thereof, and how phantom types allow us to construct type
safe OpenCL kernels in Standard ML, and a description of how the SmlCL
module was created with this functionality. Finally, we will discuss
possible additions and extensions to our work as well as related work,
followed by a conclusion.

We will also conduct a series of benchmarks on selected parts of
SmlCL, in order to compare performance with traditional sequential
code. All benchmarks and tests have been performed on a machine
equipped with an AMD Opteron 6274 processor, 128 Gb RAM and a GeForce
GTX 690 video card.
