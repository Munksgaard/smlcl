\section{Preliminary Interfacing with OpenCL from ML}

We want to facilitate an easy and safe way to take advantage of the
computational power of GPUs. By hiding some of the complexities of
parallel computations and choosing some sane default options, we hope
to make a simple interface to OpenCL that is easily usable from
Standard ML, without divulging too much of OpenCLs internal workings.

In order to achieve this, we have built a series of abstraction layers
on top of OpenCL: \emph{SimpleCL} is a C library that provides a
simplified C API to the OpenCL API, PrimCL provides a dumb interface
from \emph{MLton}, an implementation of Standard ML, to
SimpleCL. Finally, SmlCL builds on top of PrimCL, and allows us to
generate typed kernels and execute them safely.

\subsection{SimpleCL}

While we've chosen to focus our development efforts on interfacing
with OpenCL from MLton, we wanted to make it easy to use SmlCL from
other implementations of Standard ML as well. Thus, instead of
implementing a complete interface from Standard ML to the OpenCL API,
the first step was to create SimpleCL, a simplified OpenCL API in C,
that could hide some of the complexities of the OpenCL framework.

SimpleCL exposes the following functions:

\begin{itemize}
  \item \texttt{sclInit}: Returns a \texttt{simplecl\_machine},
    which is a struct containing a \texttt{cl\_machine} and a
    \texttt{cl\_context}, or \texttt{null} on error.
  \item \texttt{sclCompile}: Is used to compile source code into
    an OpenCL kernel. Returns a \texttt{simplecl\_kernel}, which is a
    struct containing a \texttt{cl\_kernel} and a
    \texttt{cl\_program}, or \texttt{null} on error.
  \item \texttt{sclCreateBuffer}: Creates a read-write buffer,
    optionally filled with the contents of \texttt{array}.
  \item \texttt{sclReadBuffer}: Read the contents of a buffer into an
    array.
  \item \texttt{sclWriteBuffer}: Write the contents of an array into a
    buffer.
  \item \texttt{sclRun1}: Run a kernel with one input buffer and one
    output buffer.
  \item \texttt{sclRun2}: Run a kernel with two input buffers and one
    output buffer.
  \item \texttt{sclFreeBuffer}: Frees a buffer from memory.
  \item \texttt{sclCleanupKernel}: Cleans up a kernel and frees it
    from memory.
  \item \texttt{sclCleanupMachine}: Cleans up a machine and frees it
    from memory.
\end{itemize}

The goal was to provide a simple interface, that allows you to perform
the basic operations required for GPU processing without having to
worry too much about the internals of the actual device. Therefore,
every function in SimpleCL is easy to use, and has a clear and well
defined purpose. For example, only one function is required to set up
the environment: \texttt{sclInit}, which automatically gets the device
and platform ID, creates the appropriate context and command queue and
returns it all in a nice struct, while checking for errors in the
function calls.

To facilitate this, \texttt{sclInit} makes some assumptions about the
intentions of the user and the programs to be executed, for example,
while OpenCL supports both CPUs, GPUs and dedicated accelerators,
SimpleCL is only targeted at GPU devices, and as a result
\texttt{sclInit} will fail, returning \texttt{NULL}, if such a device
is not found.

Similarly, \texttt{sclCompile} assumes that only one kernel is to be
compiled for each string of source code it gets. It is not possible to
create several kernels from the same program, without calling
\texttt{sclCompile} several times.

Furthermore, buffers are always created as read-write
buffers. Unfortunately, this means that, for some devices such as the
NVIDIA Kepler GK110, we will not be able to take advantage of the
built-in read-only cache, which may result in sub-optimal performance
for some programs. However, as our goal was to achieve maximum
flexibility and ease-of-use, while minimizing complexity, this was
deemed a reasonable compromise. This way, buffers can be reused
easily and kernels can both read and write to them at their leisure.

One of the other compromises that we've had to make, is that kernels
only take buffers as arguments. While we would like to enable users to
add scalar arguments to their kernels, this would add a lot of
complexity to the interface. Besides, while not nearly as efficient,
you can still provide individual arguments in buffers of size 1.

Additionally, while the run functions, \texttt{sclRun1} and
\texttt{sclRun2}, distinguish between input and output buffers by
convention (as seen by their naming schemes: \texttt{sclRun1} actually
takes 2 buffers as arguments, while \texttt{sclRun3} takes 3 buffers
as arguments), internally they are not handled differently. Combined
with the fact that all buffers are both read and write, this makes for
some quite versatile run functions, that accept any combination of
input and output buffers.

\subsection{PrimCL}

The PrimCL library provides a simple wrapper for SimpleCL using the
MLton foreign function interface~\cite{ffi}. The idea is that, to port
SmlCL to another ML implementation, you just need to provide a new
implementation of PrimCL, which SmlCL can then use to communicate with
the GPU. It also provides the constants \texttt{intSize} and
\texttt{realSize}, which corresponds to the native size, in machine
words, of ML integers and reals (doubles in OpenCL), respectively. A
port would also need to modify these to fit the architecture and
implementation of integers and reals.

With PrimCL in place, we are now ready to build another layer of
abstraction, that will provide us with a type-safe and easy to use
interface to OpenCL.
