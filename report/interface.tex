\section{Interfacing with OpenCL from ML}

We want to facilitate an easy and safe way to take advantage of the
computational power of GPUs. By hiding some of the complexities of
parallel computations and choosing some sane default options, we hope
to make a simple interface to OpenCL that is easily usable from
Standard ML, without divulging too much of OpenCLs internal workings.

In order to achieve this, we have built a series of abstraction layers
on top of OpenCL: \emph{SimpleCL} is a C library that provides a
simplified C API to the OpenCL API, the first iteration of \emph{SmlCL} then
provided a dumb interface from \emph{MLton}, an implementation of Standard
ML, to SimpleCL. Building on top of SmlCL, we've added the ability to
safely generate typed kernels and execute them in a type safe manner.

\subsection{SimpleCL}

While we've choosen to focus our development efforts on interfacing
with OpenCL from MLton, we wanted to make it easy to use SmlCL from
other implementations of Standard ML as well. Thus, instead of
implementing a complete interface from Standard ML to the OpenCL API,
the first step was to create SimpleCL, a simplified OpenCL API in C,
that could hide some of the complexities of the OpenCL framework.

In its current form, SimpleCL provides the following interface:

\begin{itemize}
  \item \texttt{sclInit}: Returns a \texttt{simplecl\_machine},
    which is a struct containing a \texttt{cl\_machine} and a
    \texttt{cl\_context}, or \texttt{null} on error.
  \item \texttt{sclCompile}: Is used to compile source code into
    an OpenCL kernel. Returns a \texttt{simplecl\_kernel}, which is a
    struct containing a \texttt{cl\_kernel} and a
    \texttt{cl\_program}, or \texttt{null} on error.
  \item \texttt{sclCreateBuffer}: Creates a read-write buffer,
    optionally filled with the contents of \texttt{array}.
  \item \texttt{sclReadBuffer}: Read the contents of a buffer into an
    array.
  \item \texttt{sclWriteBuffer}: Write the contents of an array into a
    buffer.
  \item \texttt{sclRun1}: Run a kernel with one input buffer and one
    output buffer.
  \item \texttt{sclRun2}: Run a kernel with two input buffers and one
    output buffer.
  \item \texttt{sclFreeBuffer}: Frees a buffer from memory.
  \item \texttt{sclCleanupKernel}: Cleans up a kernel and frees it
    from memory.
  \item \texttt{sclCleanupMachine}: Cleans up a machine and frees it
    from memory.
\end{itemize}
