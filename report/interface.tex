\section{Preliminary Interfacing with OpenCL from ML}

We want to facilitate an easy and safe way to take advantage of the
computational power of GPUs. By hiding some of the complexities of
parallel computations and choosing some sane default options, we hope
to provide a simple interface to OpenCL that is easily usable from
Standard ML, without divulging too much of OpenCLs inner workings.

In order to achieve this, we have built a series of abstraction layers
on top of OpenCL: \emph{SimpleCL} is a C library that provides a
simplified C API to the OpenCL API, PrimCL provides a dumb interface
from \emph{MLton}, an implementation of Standard ML, to
SimpleCL, and SmlCL builds on top of PrimCL, allowing us to
generate statically typed kernels and execute them safely.

\subsection{SimpleCL}

While we've chosen to focus our development efforts on interfacing
with OpenCL from MLton, we wanted to make it easy to use SmlCL from
other implementations of Standard ML as well. Thus, instead of
implementing a complete interface from Standard ML to the OpenCL API,
the first step was to create SimpleCL, a simplified OpenCL API in C,
that could hide some of the complexities of the OpenCL framework.

The code for SimpleCL can be seen in appendix \ref{app:simplecl}, but
in short it exposes the following functions:

\begin{description}
  \item[\texttt{sclInit}] \hfill \\
    Returns a \texttt{simplecl\_machine}
    which is a struct containing a \texttt{cl\_machine} and a
    \texttt{cl\_context}, or \texttt{null} on error.
  \item[\texttt{sclCompile}] \hfill \\ Is used to compile source code into
    an OpenCL kernel. Returns a \texttt{simplecl\_kernel}, which is a
    struct containing a \texttt{cl\_kernel} and a
    \texttt{cl\_program}, or \texttt{null} on error.
  \item[\texttt{sclCreateBuffer}] \hfill \\ Creates a read-write buffer,
    optionally filled with the contents of \texttt{array}.
  \item[\texttt{sclReadBuffer}] \hfill \\ Read the contents of a buffer into an
    array.
  \item[\texttt{sclWriteBuffer}] \hfill \\ Write the contents of an array into a
    buffer.
  \item[\texttt{sclRun1}] \hfill \\ Run a kernel with one input buffer and one
    output buffer.
  \item[\texttt{sclRun2}] \hfill \\ Run a kernel with two input buffers and one
    output buffer.
  \item[\texttt{sclFreeBuffer}] \hfill \\ Frees a buffer from memory.
  \item[\texttt{sclCleanupKernel}] \hfill \\ Cleans up a kernel and frees it
    from memory.
  \item[\texttt{sclCleanupMachine}] \hfill \\ Cleans up a machine and frees it
    from memory.
\end{description}

The goal was to provide a simple interface, that allows you to perform
the basic operations required for GPU processing without having to
worry too much about the internals of the actual device. Therefore,
every function in SimpleCL is easy to use, and has a clear and well
defined purpose. For example, only one function is required to set up
the environment: \texttt{sclInit}, which automatically gets the device
and platform ID, creates the appropriate context and command queue and
returns it all in a nice structure, while checking for errors in the
function calls.

To facilitate this, \texttt{sclInit} makes some assumptions about the
intentions of the user and the programs to be executed. For example,
while OpenCL supports both CPUs and GPUs, SimpleCL is only targeted at
GPU devices, and as a result \texttt{sclInit} will fail if no such
device is found, returning \texttt{NULL}.

Similarly, \texttt{sclCompile} assumes that only one kernel is to be
compiled for each string of source code it gets. It is not possible to
create several kernels from the same source code without calling
\texttt{sclCompile} several times.

Furthermore, buffers are always created as read-write
buffers. Unfortunately, this means that, for some devices such as the
NVIDIA Kepler GK110, we will not be able to take advantage of the
built-in read-only cache, which may result in sub-optimal performance
for some programs. However, as our goal was to achieve maximum
flexibility and ease-of-use, while minimizing complexity, this was
deemed a reasonable compromise. By making all buffer read-write, they
can be reused easily and we always know that kernels can read and
write to them at their leisure.

One of the other compromises that we've had to make is that kernels
only take buffers as arguments. While we would like to enable users to
add scalar arguments to their kernels, this would add a bit of
complexity to SimpleCL, and a lot of complexity to PrimCL. Besides,
while not nearly as efficient, you can still provide individual
arguments in buffers of size 1.

Additionally, while the run functions, \texttt{sclRun1} and
\texttt{sclRun2}, distinguish between input and output buffers by
convention (as seen by their naming schemes: \texttt{sclRun1} actually
takes 2 buffers as arguments, while \texttt{sclRun3} takes 3 buffers
as arguments), internally they are not handled differently. Combined
with the fact that all buffers are both read and write, this makes for
some quite versatile run functions, that accept any combination of
input and output buffers.

\subsection{PrimCL}

The PrimCL library provides a simple wrapper for SimpleCL using the
MLton foreign function interface. The idea is that, to port SmlCL to
another ML implementation, you just need to provide a new
implementation of PrimCL, which SmlCL can then use to communicate with
the GPU. In addition to the functions in SimpleCL, PrimCL also
provides the constants \texttt{intSize} and \texttt{realSize}, which
corresponds to the native size, in machine words, of ML integers and
reals, which are used when creating buffers.

It should be noted that initially, development on PrimCL took place
using the Moscow ML~\cite{mosml} implementation of Standard ML, but
after some struggles with the foreign function interface, it was
decided to make the switch to MLton, which has a substantially simpler
(but also less powerful) foreign function interface. MLton has the
advantages that many scalars such as reals, integers, pointers, and
characters have a simple one-to-one relation to their C equivalent
types.

Using PrimCL, we can interface with GPUs in a simple yet powerful
manner. However, PrimCL offers no guarantees that kernels executed are
correct, or will even run, and does not guarantee that kernels are
correctly initiated with the correct buffers, etc. But, with PrimCL in
place, we are now ready to build another layer of abstraction, which
will provide us with a powerful type-safe interface to OpenCL. In
order to do that, we have to introduce phantom types.
