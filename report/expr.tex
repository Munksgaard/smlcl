\section{OpenCL Kernel Generating}

We wish to avoid having to code kernels in OpenCLs C-like language. In
the implementation of SmlCL that we have described so far, kernels are
simply strings that we manually attach some types to. It is easy to
see, that this approach has the potential of introducing a great many
bugs and errors in our program. Firstly, because we have no way of
knowing whether or not the source code we read from a file is actually
valid OpenCL C or not, but also because we don't know if the types
we attach to the kernels are correct.

Both of these problems can be solved by having SmlCL generate kernels
for us, which is what we have chosen to do in SmlCL.

We have extended SmlCL with the type \texttt{$\alpha$ expr}, and the
following functions for constructing expressions, where we again make
use of phantom types to guarantee that expressions are well-typed:

\begin{itemize}
  \item \texttt{IntC}: Models an integer constant.
  \item \texttt{RealC}: Models a constant of type real.
  \item \texttt{Add}, \texttt{Sub}, \texttt{Mul}, and \texttt{Div}:
    Addition, subtraction, multiplication, and division of two expressions of the same type.
  \item \texttt{Eq}, \texttt{And}, \texttt{Or}, and \texttt{Not}:
    Equality, and the boolean operations and, or, and not.
  \item \texttt{IntToReal} and \texttt{RealToInt}: Conversion between
    types.
\end{itemize}

Note that there is seemingly no way to access the buffer-arguments to
the kernel. In the structure implementing SmlCL, you can see that
there are functions \texttt{Buf1} and
\texttt{Buf2}, which take a type variable and an index (can be either
\texttt{This}, referring to the value in the buffer corresponding to
the particular instance of the kernel that is running, through
\texttt{get\_global\_id(0)}, an absolute index into the array, or an
offset from \texttt{This}), and return an expression of the given
type. If we had simply exposed these functions to the user, he would
be able to construct badly typed expression such as \texttt{Add (Buf1
  Real This) (Buf1 Int This)}. Instead, the user is required to pass a
clojure to the kernel generating functions, where the arguments to the
clojure corresponds to a specific buffer. Here is what that looks
like:

\begin{lstlisting}[mathescape,language=ML,caption=Expressing VectorAdd
  using SmlCLs expression generating capabilities.,label=vectoradd2]
  mkKern2 machine
          "VectorAdd"
          (fn (b1, b2) => Add (b1 This) (b2 This))
          (Real, Real)
          Real;
\end{lstlisting}

Where the types for the buffers is given as the fourth and fifth
argument. This way, we are guaranteed that all references to a buffer
has the same type.

As en example of how expression generating works, here is the kernel
generated by running the code in Listing \ref{vectoradd2}:

\begin{lstlisting}[mathescape,language=C,caption=VectorAdd as
    generated by SmlCL]
  #pragma OPENCL EXTENSION cl_khr_fp64 : enable
  __kernel void VectorAdd(
  __global const int* buf1,
  __global const double* buf2,
  __global double* bufr) {
  int iGID = get_global_id(0);
  bufr[iGID] = ((real)(buf1[iGID]) + buf2[iGID]);
  }
\end{lstlisting}

As the observant reader might have noticed, right now SmlCL can only
represent a \emph{very} limited subset of C. For starters, there is no
way to represent statements og even control flow structures. In their
current form, expressions are limited to a single statement, assigning
a value to the position in the buffer corresponding to the current
global id. While allowing us to keep the expression generator rather
simple, this is obviously not ideal, and, as we'll discuss in chapter
\ref{futurework}, an obvious improvement would be to be able to
include if statements, loops, temporary variables and the
like.
