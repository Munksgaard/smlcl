\section{Type Safe OpenCL Kernel Execution}

Now that we have introduced phantom types, we are ready to describe
our implementation of SmlCL.

The first iteration of the SmlCL interface provided the
following functions:

\begin{itemize}
  \item \texttt{init}: Initiates a GPU device and returns a value of
    the type \texttt{machine}, which is used to execute commands to
    the device.
  \item \texttt{Real} and \texttt{Int}: Returns a type variable of
    type \texttt{$\alpha$ T} (instantiated with the corresponding
    types), containing closures for making, reading, and writing to
    buffers.
  \item \texttt{mkKern1} and \texttt{mkKern2}: Is used to compile
    kernels from source code in the form of a text string. The
    functions are used to compile kernels that take 1 and 2 input
    buffers as arguments, respectively, in addition to an output
    buffer. No checking is done on the source string, so this function
    relies on the source string being syntactically and semantically
    correct, while relying on additional type arguments for
    determining the type of the resulting \texttt{kern1} and
    \texttt{kern2}.
  \item \texttt{mkBuf}, \texttt{readBuf}, and \texttt{writeBuf}:
    Functions for handling buffers. We can create a new buffer
    containing the elements of an input array, read the contents of a
    buffer into an array, and write the contents of an array to an
    already existing buffer (while making sure that the input array is
    no larger than the size of the buffer).
  \item \texttt{kcall1} and \texttt{kcall2}: Executes a kernel. Takes
    a number of input buffers as arguments, and produces a resulting
    output buffer. Also takes an integer, the work size \texttt{n}, as
    argument, and assumes that the output buffer contains \texttt{n}
    elements.
\end{itemize}

The implementation of SmlCL takes advantage of phantom types to
ensure that programs are well typed by making the types for buffers
and kernels parameterized. Since they can only be instantiated using
\texttt{Real} and \texttt{Int}, which have types \texttt{real T} and
\texttt{int T}, respectively, they each get the appropriate type.

The major weak points are \texttt{mkKern1} and \texttt{mkKern2}, since
they rely on the user to provide the correct types for the source
string, as well as the source string actually being correct OpenCL C.
To help alleviate this problem, SmlCL has been extended with kernel
generating facilities.
